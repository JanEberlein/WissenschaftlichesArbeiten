\documentclass{article}
\usepackage[utf8]{inputenc}
\usepackage[ngerman]{babel}
\usepackage[style=numeric]{biblatex}
\usepackage{longtable}
\addbibresource{references.bib}

\title{WiA}
\author{Jan Eberlein}
\date{May 2022}

\begin{document}

\maketitle

\section{Laboraufgabe 7}

\subsection{Zusammenfassung und Bewertung}
\subsubsection*{Lüdecke}
Lüdecke versucht mit dem Paper zu belegen, dass der momentane Klimawandel nicht durch die Menschheit verursacht ist.
Dafür werden Temperatur-Messdaten nach stark spezifischen Parametern gefiltert und anschließend als repräsentativ behandelt.
Ziel dieses Papers scheint nicht Erkenntnisgewinn, sonder Bestätigung der eigenen politischen Agenda zu sein.
\\

\textbf{Wer sind die Autoren?}
Horst-Joachim Lüdecke, Rainer Link, and Friedrich-Karl Ewert
EIKE, European Institute for Climate and Energy

\textbf{Wo ist der Artikel publiziert worden?}
Electronic version of an article published in International Journal of Modern Physics
C, Vol. 22, No. 10, doi:10.1142/S0129183111016798 (2011)
https://www.theguardian.com/environment/climate-consensus-97-per-cent/2014/apr/18/global-warming-carbon-not-cfcs

\textbf{Wer finanziert die Arbeit?}
allgemein sehr intransparent
- 'Partner'
- 'fremde Aufträge'
- Spenden 
https://eike-klima-energie.eu/ueber-uns/

\textbf{Wie wird mit unsicheren Erkenntnissen umgegangen?}
Nicht erkennbar, irgendwas mit Konfidenzintervall, aber intransparent
(Lüdecke et al.: Conclusion; IPCC: Box “Treatment of uncertainty”, S.27)

\textbf{Welche der wissenschaftlichen Normen werden von Lüdecke nicht eingehalten?}
ERGEBNISOFFENHEIT

Uneigennützigkeit: Arbeit für stark politische Organisiation

Skeptizismus: Ergebnisse werden nicht kritisch reflektiert.

(Webseite von EIKE – Über uns – Die Mission – Grundsatzpapier Klima -
https://eike-klima-energie.eu/die-mission/grundsatzpapier-klima/)

\textbf{Auf welchen (wie vielen) Daten basierten die Aussagen?}
"2249 worldwide monthly temperature records from GISS (NASA) with the 100-year
period covering 1906-2005 and the two 50-year periods from 1906 to 1955
and 1956 to 2005" aber sehr stark gefiltert
(Lüdecke et al.: Abstact, Tabelle 1; IPCC: Introduction, Chapter 1 "Observed
changes in climate and their effects", S. 2-4)


\subsubsection*{IPCC}
Der IPCC Climate Change 2007 Synthesis Report dient als Zusammenfassung mehrerer langjährigen Studien, die sich mit Ursachen, Auswirkungen und Milderung von Klimawandel auseinander setzten.
Besonderer Fokus liegt auf sozio-ökonomischen Folgen und politischen Handlungsempfehlungen.
\\

\textbf{Wer sind die Autoren?}
Published by the Intergovernmental Panel on Climate Change

Core Writing Team:
Lenny Bernstein, Peter Bosch, Osvaldo Canziani, Zhenlin Chen, Renate Christ, Ogunlade Davidson,
William Hare, Saleemul Huq, David Karoly, Vladimir Kattsov, Zbigniew Kundzewicz, Jian Liu, Ulrike
Lohmann, Martin Manning, Taroh Matsuno, Bettina Menne, Bert Metz, Monirul Mirza, Neville Nicholls,
Leonard Nurse, Rajendra Pachauri, Jean Palutikof, Martin Parry, Dahe Qin, Nijavalli Ravindranath,
Andy Reisinger, Jiawen Ren, Keywan Riahi, Cynthia Rosenzweig, Matilde Rusticucci, Stephen Schneider, Youba Sokona, Susan Solomon, Peter Stott, Ronald Stouffer, Taishi Sugiyama, Rob Swart,
Dennis Tirpak, Coleen Vogel, Gary Yohe (siehe Annex IV für detailliertere Auflistung)

Technical Support Unit for the Synthesis Report:
Andy Reisinger, Richard Nottage, Prima Madan

\textbf{Wo ist der Artikel publiziert worden?}
IPCC

\textbf{Wer finanziert die Arbeit?}
IPCC \verb|->| UNEP, WMO

\textbf{Wie wird mit unsicheren Erkenntnissen umgegangen?}
drei verschiedene Methoden für Konfidenz-Bewertung, je nach Datenart
Herkunft und Art der Bewertung ist nachvollziehbar
(Lüdecke et al.: Conclusion; IPCC: Box “Treatment of uncertainty”, S.27)

\textbf{Auf welchen (wie vielen) Daten basierten die Aussagen?}
ja

citing over 6000 peer-reviewed scientific studies 
\\\verb|https://en.wikipedia.org/wiki/IPCC_Fourth_Assessment_Report|

Jede der drei Arbeitsgruppen-Berichte verwaltet die genutzten Referenzen kapitelweise, jeweils mehrere Seiten Quellen

Art: Studien zu granulierten Subthemen der jeweiligen Arbeitsgruppen bzw. Themen

\pagebreak
\section{Laboraufgabe 8}

\renewcommand*{\arraystretch}{1.5}
\begin{longtable}{@{}|p{0.2\textwidth}|p{0.4\textwidth}p{0.4\textwidth}|@{}}
    \hline
                      & spezielle Relativität                                                         & Lüdecke                                                          \\ \hline
    \endfirsthead
    %
    \endhead
    %
    \hline
    \endfoot
    %
    \endlastfoot
    %
                      & theoretische Physik                                                           & angewande Physik                                                 \\
                      & (verzicht auf Hilfs-Hypothesen)                                               & intransparente und z.T. falsche Annahmen                         \\
                      & findet Erklärungen für viele beobachtbare Phänomene                           & ignoriert die Realität                                           \\
                      & teilt Ansätze (bzw. hat viele wie) andere Menschen in der führenden Forschung & defamiert Gegenposition                                          \\
                      & behandelt andere Theorien und Positionen neutral                              & falsche Anschuldigungen und Polemik                              \\
    Auswirkungen      & Nachbesserung der Ausarbeitung bei Kritik                                     & Reviews finden multiple Fehler -\textgreater Nutzung von Fake-Journalen      \\
    Lebenslauf        & Während Promotion                                                             & nach Pensonierung                                                \\
    Ausbildung        & "Fachlehrer (Diplom) in mathematischer Richtung" (Physik)                     & Physik-Studium                                                   \\
    Veröffentlichung  & Annalen der Physik und Chemie (passt thematisch)                              & International Journal of Modern Physics (passt nicht thematisch) \\
    Interessen-Gruppe & {[}keine gefunden{]}                                                          & politisch! (vlt wirtschaftlich)                                  \\ \hline
    \end{longtable}

\end{document}
